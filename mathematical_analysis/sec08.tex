\section{Непрерывность функции. Точки разрыва}

\begin{definition}
  Функция $f(x)$, определённая в некоторой окрестности точки $x_0$, называется непрерывной в этой точке если: \[
    \exists \lim_{x \to x_0} f(x) = f(x_0)
  \]
\end{definition}

Множество непрерывных функций в точке $x_0$ обозначается $C(x_0)$\[
  f(x) \in C(x_0) \iff \text{ - функция непрерывна в точке } x_0
\] 

\begin{eg}
  \[
  \lim_{x \to 0} \sin(x) = \sin(x) = 0 \iff \sin(x) \in C(0)
  \] 
\end{eg}
\begin{eg}
  \[
  sgn x = \begin{cases}
    1, x > 0 \\
    0, x = 0 \\
    -1, x < 0
  \end{cases} \implies
  sgn \not \in C(0)
  /]
\end{eg}

\begin{definition}
  Функция $y = f(x)$, определённая в некоторой окрестности точки $x_0$, называется непрерывной в этой точке, если в достаточно малой окрестности точки $x_0$ значение функции близки к $f(x_0)$.
  \begin{gather*}
    y = f(x) \in C(x_0) \\
    \iff \\
    (\forall  \varepsilon > 0)(\exists \delta(\varepsilon) > 0)(\forall x \in \mathring{S}(x_0, \delta) \implies |f(x) - f(x_0)| < \varepsilon)
  \end{gather*}
\end{definition}

\begin{definition}
  Функция $y = f(x)$ в некоторой окрестности точки $x_0$ называется непрерывной в этой точке, если выполняются условия:
  \begin{align*}
    &1. \quad \exists \lim_{x \to x_0+} f(x) \\
    &2. \quad \exists \lim_{x \to x_0-} f(x) \\
    &3. \quad \lim_{x \to x_0+} f(x) = \lim_{x \to x_0-} f(x) = f(x) \\
  \end{align*}
\end{definition}

Пусть $y = f(x)$ определена в некоторой точке в окрестности $x_0$.
Выберем произвольный $x$ в этой окрестности.
Тогда:
\begin{align*}
  &\boxed{\Delta x = x - x_0} \text{ - приращение функции} \\
  &\boxed{\Delta y = f(x) - f(x_0)} \text{ - соответствующее приращение функции}
\end{align*}

% Добавь наконец шорткаты для align-а!
% и для mathring-а тоже

\begin{definition}
  Функция $y = f(x)$ называется непрерывной в точке $x_0$, если бесконечно малому приращению аргумента соответствует бесконечно малое приращение функции. \[
    \lim_{\Delta x \to 0} \Delta y = 0
  \] 
\end{definition}

\subsection{Односторонняя непрерывность}

\begin{definition}
  Функция $y = f(x)$ определённая в правосторонней окрестности точки $x_0$ (математическим языком - $[x_0, x_0 + \delta)$) называется непрерывной справа в этой точке, если: \[
  \exists \lim_{x \to x_0+} = f(x_0)
  \] 
\end{definition}

\begin{definition}
  Функция $y = f(x)$ определённая в левосторонней окрестности точки $x_0$ (математическим языком - $(x_0 - \delta, x_0]$) называется непрерывной справа в этой точке, если: \[
  \exists \lim_{x \to x_0-} = f(x_0)
  \] 
\end{definition}

\begin{theorem}
  Для того, чтобы функция $y = f(x)$ была непрерывна в точке $x_0$ необходимо и достаточно, чтобы она была непрерывна в этой точке справа и слева. 
\end{theorem}

\begin{definition}
  Функция $y = f(x)$ называется непрерывной на интервале $(a, b)$, если она непрерывна в каждой точке этого интервала.
\end{definition}

\begin{definition}
  Функция $y = f(x) $ называется непрерывной на отрезке $[a, b]$, если:
  \begin{enumerate}
    \item Непрерывна на интервале $(a, b)$
    \item Непрерывна в точке $a$ справа
    \item Непрерывна в точке $b$ слева
  \end{enumerate}
\end{definition}

\begin{itemize}
  \item $C(a, b)$ - множество функций, непрерывных на интервале. 
  \item $C[a, b]$ - множество функций, непрерывных на отрезке. 
  \item $C(X)$ - множество функций, непрерывных на промежутке $X$. 
\end{itemize}

\subsection{Классификация точек разрыва}

\begin{definition}
  Пусть функция $y = f(x)$ определена в некоторой точке проколотой окрестности точки $x_0$ непрерывна в любой точке этой окрестности (за исключением самой точки $x_0$).
  Тогда точка $x_0$ называется точкой разрыва функции.
\end{definition}

% Переверстать
Пусть точка $x_0$ - точка разрыва. Её можно классифицировать как:
\begin{itemize}
  \item I-ого рода
    \begin{itemize}
    \item Основное условие \[
      \exists \lim_{x \to x_0 +-}
\]
    \item Точка конечного разрыва \[
      \lim_{x \to x_0+} \neq \lim_{x \to x_0-}   
    \] 
    \item Точка устранимого разрыва \[
      \lim_{x \to x_0+} = \lim_{x \to x_0-} \neq f(x_0) \text{ или } \not\exists f(x_0)   
    \]  
  \end{itemize}
  \item II рода \[
  \not \exists \lim_{x \to x_0+-} 
  \] 
\end{itemize}

\begin{definition}
  Если точка $x_0$ -- точка разрыва функции $y = f(x)$ и существуют конечные пределы $\lim_{x \to x_0+} f(x)$ и $\lim_{x \to x_0-} f(x)$, то $x_0$ называют точкой I-го рода.
\end{definition}

\begin{definition}
  Если точка $x_0$ -- точка разрыва функции $y = f(x)$ и \textbf{не} существуют конечные пределы $\lim_{x \to x_0+} f(x)$ и $\lim_{x \to x_0-} f(x)$ или $\lim_{x \to x_0} f(x) = \infty$, то $x_0$ называется точкой разрыва II-го рода.
\end{definition}

\begin{definition}
  Если точка $x_0$ -- точка разрыва первого рода функции $y = f(x)$, и предел $\lim_{x \to x0+} f(x) \neq \lim_{x \to x_0-} f(x)$, то $x_0$ называется точкой конечного разрыва или точкой \textit{скачка}.
\end{definition}

\begin{definition}
  Если точка $x_0$ -- точка разрыва первого рода функции $y = f(x)$, и предел  $\lim_{x \to x_0+} f(x) = \lim_{x \to x_0-} f(x)$, но $\not\exists f(x_0)$, то точка $x_0$ называется точкой устранимого разрыва.
\end{definition}

\subsubsection*{Примеры}

\begin{eg}
  \begin{gather*}
    y = \frac{|x - 1|}{x - 1} \\
    D_f = \R \setminus \{1\} \\
    x = 1 \text{ - точка разрыва} \\
  \lim_{x \to 1+} f(x) = \lim_{x \to 1+} \frac{|x - 1|}{x - 1} = \frac{x - 1}{x - 1} = 1 \\
  \lim_{x \to 1-} f(x) = \lim_{x \to 1-} \frac{|x - 1|}{x - 1} = \frac{1 - x}{x - 1} = -1 \\
  \lim_{x \to 1+} f(x) \neq \lim_{x \to 1-} f(x) \\
  \implies x = 1 \text{ - т.р. I рода, точка скачка}\\
  \Delta f = |\lim_{x \to 1+} f(x) - \lim_{x \to 1-} f(x)| = |1 - (-1)| = 2  
  \end{gather*}
\end{eg}

% Добавить шорткаты для Df, Dx, Dy

\begin{eg}
  \begin{gather*}
    y = \frac{\sin(x)}{x} \\
    D_f = \R \setminus \{0\} \\
    \lim_{x \to 0+} f(x) = \lim_{x \to 0+} \frac{\sin(x)}{x} = 1 \\
    \lim_{x \to 0-} f(x) = \lim_{x \to 0-} \frac{\sin(x)}{x} = 1 \\
    \lim_{x \to 0+} f(x) = \lim_{x \to 0-} f(x) \\
    \implies x = 0 - \text{ т.р. I рода, устранимая точка разрыва} \\
    g(x) = \begin{cases}
      \frac{\sin(x)}{x}, x \neq 0 \\
      1, x = 0
    \end{cases} \\
    f(x) \not\in C(0) \\
    g(x) \in C(0)
  \end{gather*}
\end{eg}

\begin{eg}
  \begin{gather*}
    y = e^{\frac{1}{x}} \\
    D_f = \R \setminus \{0\} \\
    \lim_{x \to 0+} f(x) = \lim_{x \to 0+} e^{\frac{1}{x}} = e^{+\infty} = \infty \\ 
    \lim_{x \to 0-} f(x) = \lim_{x \to 0-} e^{\frac{1}{x}} = e^{-\infty} = 0 \\ 
    \lim_{x \to 0+} f(x) = \infty \\
    \implies x = 0 \text{ - т.р. II рода}
  \end{gather*}
\end{eg}

\subsection{Свойства непрерывных функций в точке}

\begin{theorem}
  Пусть функции: \[
    \begin{rcase}
      y = f(x) \\
      y = g(x)
    \end{rcase}
    \in C(x_0)
  \] 
  Тогда:
  \begin{gather*}
    f(x) + g(x) \in C(x_0) \\
    (f \cdot g)(x) \in C(x_0)
  \end{gather*}
\end{theorem}
\begin{proof}
  По определению непрерывной функции: 
  \begin{gather*}
    \lim_{x \to x_0} f(x) = f(x_0) \\
    \lim_{x \to x_0} g(x) = g(x_0) \\
  \end{gather*}
  Рассмотрим:
  \begin{gather*}
    \lim_{x \to x_0} (f(x) + g(x)) = \lim_{x \to x_0} f(x) + \lim_{x \to x_0} g(x) + f(x_0) = g(x_0) \\
    \implies f(x) + g(x) \in C(x_0) \\
    
    \lim_{x \to x_0} (f \cdot g)(x) = \lim_{x \to x_0} f(x) \cdot g(x) = \lim_{x \to x_0} f(x) \cdot \lim_{x \to x_0} g(x) = f(x_0) \cdot g(x_0) \\
    \implies (f \cdot g)(x) \in C(x_0)

    \lim_{x \to 0} \frac{f(x)}{g(x)} = \frac{\lim_{x \to x_0} f(x)}{\lim_{x \to x_0} g(x_0}
  \end{gather*}
\end{proof}

