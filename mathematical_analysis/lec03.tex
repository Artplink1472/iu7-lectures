\subsection{Свойства сходящихся последовательность}

\begin{theorem}
  \textit{О существовании единственности предела последовательности} \\
  Любая сходящаяся последовательность имеет единственный предел.
\end{theorem}
\begin{proof} Аналитическое доказательство.
  Пусть $\{x_{n}\} $ - сходящаяся последовательность. \\
  Рассуждаем методом от противного. Пусть последовательность $\{x_{n}\} $ более одного предела.
  \begin{gather*}
    \lim_{n \to \infty} = a \\
    \lim_{n \to \infty} = b \\
    a \neq b \\
  \end{gather*}
  \begin{gather}
    \lim_{n \to \infty} = a \iff (\forall \epsilon_1 > 0)(\exists N_1(\epsilon_1) \in N)(\forall n > N_1(\epsilon_1) \implies |x_{n} - a| < \epsilon_1) \\
    \lim_{n \to \infty} = b \iff (\forall \epsilon_2 > 0)(\exists N_2(\epsilon_2) \in N)(\forall n > N_2(\epsilon_2) \implies |x_{n} - b| < \epsilon_2)  
  \end{gather} 
  Выберем $N=max \{N_1\left( \epsilon_1 \right) , N_2\left( \epsilon_2 \right) \}$. \\
  Пусть 
  \[
    \epsilon_1 = \epsilon_2 = \epsilon = \frac{|b - a|}{3}
  \]
  \begin{gather*}
  3 \epsilon = |b - a| = |b - a + x_{n} - x_{n}| = \\
  = |(x_{n} - a) - (x_{n} - b)| \le |x_{n} - a| + |x_{n} - b| < \epsilon_1 + \epsilon_2 = 2 \epsilon \\
  3 \epsilon < 2 \epsilon
  \end{gather*}
  Противоречие. Значит, предоположение не является верным $\implies$ последовательность $x_{n}$ имеет единственный предел.
\end{proof}

\begin{proof} Геометрическое доказательство\\
  Нельзя уложить бесконечное число членов последовательности $x_{n}$ в две непересекающиеся окрестности. \\
  % рисуночек
\end{proof}

\begin{theorem}
  \textit{Об ограниченности сходящейся последовательности}. \\ 
  Любая сходящаяся последовательность \textit{ограничена}. 
\end{theorem}

\begin{proof}
  По определению сходящейся последовательности 
  \begin{gather*}
    \implies \lim_{n \to \infty} = a \iff (\forall \epsilon > 0)(\exists N(\epsilon)\in \N)(\forall n > N(\epsilon) \implies |x_{n} - a| < \epsilon).
  \end{gather*}
  Выберем в качестве $M = max \{|x_{1}|, |x_2|, \ldots |x_n|, |a - \epsilon|, |a + \epsilon|\}$. \\
Тогда для $\forall n \in \N$ будет верно $|x_{n}| \le M$ - это и ознaчает, что последовательность $x_{n}$ - ограниченная.
\end{proof}

\begin{theorem}
  \textit{Признак сходимости Вейерштрасса}. \\
  Ограниченная монотонная последовательность сходится.
\end{theorem}

\subsubsection{Предел последовательности x_{n} = \left( 1 + \frac{1}{n} \right)}

\begin{theorem}
  Последовательность $x_{n} = \left( 1 + \frac{1}{n} \right) $ имеет предел равный $e$.
  \[
  \lim_{n \to \infty} \left( 1 + \frac{1}{n} \right) = e 
  \] 
\end{theorem}

\section{Предел функции}

\begin{definition}
  Окрестностью, из которой исключена точка $x_{0}$ называется \texbf{проколотой окрестностью}.
  \[
    \mathring{S}(x_0; \delta) = S(x_0; \delta) \setminus {x_0}
  \] 
\end{definition}

\begin{definition}
  \textit{Определение функции по Коши} или \textit{на языке $\epsilon$ и $\delta$ }. \\
  Число $a$ называется пределом функции $y = f\left( x \right) $ в точке $x_0$, если $\forall \epsilon > 0$ найдется $\delta$, зависящее от  $\epsilon$ такое что $\forall x \in \mathring{S}(x_0; \delta)$ будет верно неравенство $|f\left( x \right) - a| < \epsilon$.
  \[
    \lim_{x \to x_0} f(x) = a \iff (\forall \epsilon > 0)(\exists  \delta(\epsilon) > 0)(\forall  x \in \mathring{S}(x_0; \delta) \implies |f(x) - a| < \epsilon)
  \]

  Эквивалентные записи определения

  \begin{gather*}
    \ldots \forall x \in \mathring{S}(x_0; \delta) \implies \ldots \\
    \ldots \forall x \neq x_0, |x - x_0| < \epsilon \implies \ldots \\
    \ldots \forall x, 0 < |x - x_0| < \delta \implies \ldots \\
  \end{gather*}
  
  \begin{gather*}
    \ldots \implies |f(x) - a| < \epsilon \\
    \ldots \implies f(x) \in \mathring{S}(a, \epsilon) \\
  \end{gather*}
\end{definition}

\subsubsection*{Геометрический смысл предела функции}

% График пярмой y=f(x)
% Две окрестности - эпсилон и дельта

Если для $\forall \mathring{S}(a; \epsilon$) найдется $\mathring{S}(x_0; \delta)$, то соответствующее значение функции лежат в  $\mathring{S}(a; \epsilon)$ (полоса $2 \epsilon$):
\[
  \forall x_1 \in \mathring{S}(x_0; \delta) \implies |f(x_1) - a| < \epsilon
\] 
\begin{definition}
  \textit{Определение предела функции по Гейне} или \textit{на языке последовательностей}. \\
  Число $a$ называется пределом $y = f\left( x \right) $ в точке $x_0$, если эта функция определена в окрестности точки $a$ и $\forall$ последовательнсти $x_{n}$ из области определения этой функции, сходящейся к $x_0$ соответствующая последовательность функций $\{f(x_{n})\}$ сходится к $a$.
  \[
  \lim_{x \to x_0} = a \iff (\forall x_{n}\in D_f)(\lim_{n \to \infty} x_{n} = x_0 \implies \lim_{n \to \infty} f(x_{n}) = a) 
  \] 
\end{definition}


\subsubsection*{Геометрический смысл}

% График прямой y=f(x)
% Две окрестности - эпсилон и дельта
% Набор исков и набор игреков на осях
\[
\forall x_{n} \lim_{n \to \infty} x_{n} = x_0
\] 
Для любых точек $x$, достаточно близких к точке $x_0$ (на языке математики $\lim_{n \to \infty} x_{n} = x_0$) соответствующие значения $f(x_{n})$ достаточно близко расположены к $a$ (на языке математики - $\lim_{n \to \infty} f(x_{n}) = a$)

\begin{theorem}
  Определение предела функции по Коши и по Гейне \textit{эквивалентны}. 
\end{theorem}

\subsection{Ограниченная функция}

\begin{definition}
  Функция называется \textbf{ограниченной} в данной области изменения аргумента $x$, если $\exists M\in \R, M > 0, |f(x)| \le M$.
\end{definition}

Если $\not \exists M \in \R, M > 0$, то функция $f(x)$ называется \textit{неограниченной}.

\begin{definition}
  Функция называется \textbf{локально ограниченной} при $x \to x_0$, если существует проколотая окрестность с центром в точке $x_0$, в которой данная функция ограничена.
\end{definition}
\begin{eg}
  \[
  y = \sin(\frac{1}{x})
  \] 
  % Построить график в MatLab-е
  В данном случае:
  \begin{gather*}
    |\sin(\frac{1}{x}) \le 1, M = 1 \\
    при x \in (-\frac{1}{\pi}; 0) \cup (0; \frac{1}{\pi})
  \end{gather*}
\end{eg}

\section{Основные теоремы о пределах}

\begin{theorem}
  \textit{О локальной ограниченности функции, имеющей предел}. \\
  Функция, имеющая конечный предел, локально ограничена.
\end{theorem}

