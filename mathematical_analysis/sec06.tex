\section{Односторонние пределы}

\begin{definition}
  Число $A_1$ называется пределом функции $y = f(x)$ в точке $x_0$ \textbf{слева}, если: \[
  \lim_{x \to x_0-} f(x) = A_1 \iff 
  (\forall \varepsilon > 0)(\exists \delta > 0)(\forall x \in (x_0 - \delta, x_0) \implies |f(x) - A_1| < \varepsilon)
  \] 
\end{definition}

\begin{definition}
  Число $A_2$ называется пределом функции $y = f(x)$ в точке $x_0$ \textbf{справа}, если: \[
  \lim_{x \to x_0+} f(x) = A_2 \iff
  (\forall \varepsilon > 0)(\exists \delta > 0)(\forall x \in (x_0, x_0 + \delta) \implies |f(x) - A_2| < \varepsilon)
  \] 
\end{definition}

Пределы справа и слева называют \textit{односторонними пределами}.

\begin{theorem}
  \textit{О существовании предела функции в точке}. \\
  Функция $y = f(x)$ в точке  $x_0$ имеет конечный предел тогда и только тогда, когда существуют пределы справа и слева и они равны между собой.
  \begin{gather*}
    \lim_{x \to x_0} f(x) = \lim_{x \to x_0+} f(x) = \lim_{x \to x_0-} f(x) 
  \end{gather*}
\end{theorem}

\subsection{Пределы на бесконечности}

\begin{definition}
  Число $a$ называется пределом функции $y = f(x)$ при $x \to + \infty$, если: \[
  \lim_{x \to +\infty} f(x) = a \iff (\forall \varepsilon > 0)(\exists N(\varepsilon) > 0)(\forall x > N) \implies |f(x) - a| < \varepsilon)
  \] 
  где $N$ - большое число, $N > 0, N \in \R$.
\end{definition}

\begin{definition}
  Число $a$ называется пределом функции $y = f(x)$ при $x \to - \infty$, если: \[
  \lim_{x \to -\infty} f(x) = a \iff (\forall \varepsilon > 0)(\exists N(\varepsilon) > 0)(\forall x < -N) \implies |f(x) - a| < \varepsilon)
  \] 
  где $N$ - большое число, $N > 0, N \in \R$.
\end{definition}

\begin{note}
 \begin{gather*}
  \lim_{x \to +\infty} f(x) = a \iff \\
  (\forall \varepsilon > 0)(\exists N(\varepsilon) > 0)(\forall x > N) \implies |f(x) - a| < \varepsilon)

  \lim_{x \to -\infty} f(x) = a \iff \\
  (\forall \varepsilon > 0)(\exists N(\varepsilon) > 0)(\forall x < -N) \implies |f(x) - a| < \varepsilon)

  \lim_{x \to \infty} f(x) = a \iff \\
  (\forall \varepsilon > 0)(\exists N(\varepsilon) > 0)(\forall x \in |x| > N) \implies |f(x) - a| < \varepsilon)

\end{gather*} 
\end{note}

\subsection{Бесконечные пределы}

\begin{definition}
  Функция $y = f(x)$ имеет бесконечный предел при $x \to x_0$, если: \[
  \lim_{x \to x_0} f(x) = \infty \iff
  (\forall M > 0)(\exists \delta(M) > 0)(\forall x \in \mathring{S}(x_0, \delta) \implies |f(x)| > M)
  \] 
  где $M$ - большое число, $M > 0, M \in \R$,
  а $\delta$ - малое число.
\end{definition}
\begin{note}
  \begin{gather*}
    \lim_{x \to x_0} f(x) = +\infty \iff
    (\forall M > 0)(\exists \delta(M) > 0)(\forall x \in \mathring{S}(x_0, \delta) \implies f(x) > M)

    \lim_{x \to x_0} f(x) = -\infty \iff
    (\forall M > 0)(\exists \delta(M) > 0)(\forall x \in \mathring{S}(x_0, \delta) \implies f(x) < -M)
  \end{gather*}
\end{note}
\begin{eg}
  \begin{gather*}
    y = \arctg(x), \qquad x \to \infty \\
    \lim_{x \to +\infty} \arctg(x) = \frac{\pi}{2} \\
    \lim_{x \to -\infty} \arctg(x) = -\frac{\pi}{2} \\
  \end{gather*}  
\end{eg}
\begin{eg}
  \begin{gather*}
    y = \ln(x), \qquad x \to 0 \\
    \lim_{x \to 0-} = \not\exists  \\
    \lim_{x \to 0+} = -\infty
  \end{gather*}
\end{eg}
\begin{eg}
  \begin{gather*}
    y = \sqrt{-x}, \qquad x \to 0 \\
    \lim_{x \to 0+} = \not\exists \\
    \lim_{x \to 0-} = 0
  \end{gather*}
\end{eg}
\begin{eg}
  \begin{gather*}
    y = \frac{1}{|x - 2|}, \qquad x \to 2 \\
    \lim_{x \to 2+} \frac{1}{|x - 2|} = +\infty \\
    \lim_{x \to 2-} \frac{1}{|x - 2|} = -\infty \\
  \end{gather*}
\end{eg}

\begin{definition}
  Функция $y = f(x)$ называется \textbf{бесконечно большой функцией} (далее - \textbf{б.б.ф.} если: \[
  \lim_{x \to x_0} f(x) = \infty
  \] 
\end{definition}

\subsubsection*{Бесконечный предел на бесконечности}
\[
  \lim_{x \to \infty} = \infty \iff (\forall M > 0)(\exists N(M) > 0)(\forall x \in |x| > N \implies |f(x)| > M) 
\] 
\subsection{Сравнение бесконечно малых и бесконечно больших функцих}

\begin{theorem}
  \textit{О связи бесконечно малой и бесконечно большой функции}. \\
  Если $\alpha(x)$ - бесконечно большая функция при $x \to x_0$, то $\frac{1}{\alpha(x)}$ - бесконечно малая функция при $x \to x_0$.
\end{theorem}
\begin{proof}
  По условию $\alpha(x)$ - б.б.ф при $x \to x_0$. По определению:
  \begin{gather*}
    \lim_{x \to x_0} \alpha(x) = \infty \iff (\forall M > 0)(\exists \delta(M) > 0)(\forall x \in \mathring{S}(x_0, \delta) \implies |f(x)| > M)
  \end{gather*}

  Рассмотрим неравенство: \[
    |\alpha(x)| > M, \forall x \in \mathring{S}(x_0, \delta)
  \]

  Обозначим $\varepsilon = \frac{1}{M}$.
  \begin{gather*}
    |\alpha(x) > M| \implies \frac{1}{|\alpha(x)|} < \frac{1}{M} \\
    \implies |\frac{1}{\alpha(x)}| < \frac{1}{M} < \varepsilon \\
  \end{gather*}

  В итоге получаем:
  \begin{gather*}
    \forall x \in \mathring{s}(x_0, \delta) \implies |\frac{1}{\alpha(x)}| < \varepsilon 
  \end{gather*}

  Что по определению является бесконечно малой функцией.
\end{proof}

\subsubsection*{1-ый замечательный предел}

\begin{theorem}
  \[
  \lim_{x \to 0} \frac{\sin(x)}{x} = 1
  \]  
\end{theorem}
\begin{proof}
  Рассмотрим $\lim_{x \to 0+} \frac{\sin(x)}{x} = 1$. Потом $\lim_{x \to 0-} \frac{\sin(x)}{x} = 1$. 
  
  Пусть $\alpha$ - угол в радианах, $x \to 0, x \in (0, \frac{\pi}{2})$.

  Тут должен быть рисунок, но его пока нет :(.

  Окружность $R = 1$.\\
  Отложим луч $OK$ под углом к оси $oX$ равным $x$, где $O(0, 0), K \in $ окружности. \\
  $KH \perp OA$.

  Рассмотрим $\triangle OKH$. $OA = 1$ как радиус. $\sin(x) = \frac{KH}{OA} = KH$. \\
  Рассмотрим $\triangle OLA$. $OA = 1$ как радиус. $\tg(x) = \frac{LA}{OA} = LA$. \\
  Из геометрических построений (да будут они когда-нибудь\ldots): \[
    S_{\triangle OKA} < S_{sec OKA} < S_{\triangle OLA}
  \] 
  \begin{gather*}
    S_{\triangle OKA} = \frac{1}{2} OA \cdot KH = \frac{1}{2} \sin(x) = \frac{\sin(x)}{2} \\
    S_{sec OKA} = \frac{1}{2} OA \cdot OK \cdot KA = \frac{1}{2} \cdot x = \frac{x}{2} \\
    S_{\triangle OLa} = \frac{1}{2} OA \cdot LA = \frac{1}{2} \cdot 1 \cdot \tg(x) = \frac{tg(x)}{2}
  \end{gather*}
  \begin{gather*}
    \frac{\sin(x)}{2} < \frac{x}{2} < \frac{\tg(x)}{2} \quad | \cdot 2 \\
    \begin{rcases}
      \sin(x) < x < tg(x) \\
      x \to 0+ \implies \begin{cases}
        \sin(x) > 0 \\
        \tg(x) > 0
      \end{cases}
    \end{rcases} \implies 
    \sin(x) < x < \tg(x) \quad | : \sin(x) \\
    1 < \frac{x}{\sin(x)} < \frac{1}{\cos(x)} \\
    \cos(x) < \frac{\sin(x)}{x} < 1
  \end{gather*}
   По теореме о предельном переходе в неравенстве: \[
   \lim_{x \to 0+} \cos(x) \le \lim_{x \to 0+} \frac{\sin(x)}{x} \le 1
   \] 

   По теореме о промежуточной функции: \[
    \lim_{x \to 0} \cos(x) = 1 \implies \lim_{x \to 0+} \frac{\sin(x)}{x} = 1 
   \] 

   Аналогично для $\lim_{x \to 0-} \frac{\sin(x)}{x} = 1$. Т.к. односторонние пределы равны: \[
   \lim_{x \to 0+} \frac{\sin(x)}{x} = \lim_{x \to 0-} \frac{\sin(x)}{x} = 1 \iff \lim_{x \to x_0} \frac{\sin(x)}{x} = 1
   \] 
\end{proof}

