\lesson{1}{История России}

\section{Древняя Русь}

Славяне точно существуют с первой половины 1-го тысячелетия.

Славяне разделились на три группы:
\begin{itemize} 
  \item Восточные славяне - Древняя Русь (Киевская Русь)
  \item Западные славяне - Польша и Чехия
  \item Южные славяне - под властью Византии
\end{itemize}

Народное собрание = Вече.
Народное ополчение - "войско" племени.

Бояре - крупные землевладельцы.

15\% мужского населения отбирают в \textit{дружину} - аналог постоянной армии.

Т.к. возникла необходимость распределения ресурсов, возникает \textit{государство}.

С возникновением государства тесно связано возникновение \textit{письменности}.

К середине 9 века возникают два сильных княжества - Новгородское и Киевское.

862 г. - Приход Рюрика к правлению.

892 г. - возникновение Киевской Руси в результате объединения Новгородского и Киевского княжеств князем Олегом.

\textbf{Норманская теория} гласит, что "русские не могут управлять сами собой, поэтому потребовались управленцы извне".

Возникновение норманской теории - результат политической деятельности Анны Иоановны для оправдания нахождения у власти иностранцев.

\subsection{Внешняя и внутренняя политика правителей Древней Руси}

\subsubsection*{Олег Вещий}

Удалось разбить хазар, перестали платить дань. \\
Мирный договор с Византией. \\
К этому времени на Руси была:
\begin{itemize}
  \item Письменность
  \item Законы (возможно, в устной форме)
  \item Религия (язычество)
\end{itemize}

Результатом мирного договора была беспошлинная торговля Руси с Византией.

\subsubsection*{Игорь Рюрикович}

После смерти Олега Византия перестала соблюдать мирный договор. \\
Следующий поход Игоря на Византию окончился неудачно. \\
Игоря убили древляне во время попытки повторной сбора дани.

\subsubsection*{Ольга Святая}

Ольга "отомстила" за смерть Игоря древлянам.
Ввела понятия \textit{уроки} - сколько дани собирать, и \textit{погосты} - места сбора дани. 
Чтобы сохранить дружеские отношения с Византией, была крещена (поэтому и "Святая").

\subsubsection*{Святослав Игоревич}

Вошёл в историю как \textit{князь-воин}. Был успешным полководцем, множество завоеваний. Был убит печенегами по дороге в Киев.

\subsubsection*{Ярополк}

Ярополк убил своего брата, Олега, претендующего на престол.

\subsubsection*{Владимир Красно Солнышко}

Владимир победил в борьбе за власть между братьями. Предотвратил набеги печеногов. \\
Продолжил дело Ольги: избавившись от всех несогласных князей, тем самым объединив все княжества единому роду. 

\textbf{Языческая религия} - религия одного народа (а не многобоженство!).

Провёл реформу религии, чтобы укрепить свою власть. Была выбрана единая религия - \texit{православие}, она была наиболее выгодной Владимиру.
С появлением религии, появляется и \textit{кириллица}. 

\subsubsection*{Святополк Окаянный}

Убил двух своих братьев, за что и получил своё прозвище.

\subsubsection*{Ярослав Мудрый}

Свергает Святополка. \\
Выдает письменный свод законов - \textbf{Русская Правда}.
Согласно первой части Русской правды, можно косвенно сказать, что все люди - теперь собственность князя. 

\textbf{Вотчина} - земля бояр.

Структура общества:
\begin{enumerate}
  \item Князь
  \item Бояре (землевладельцы)
  \item Смерды (городские)
  \item Людин (крестьяне)
  \item Холопы (рабы)
  \item Изгои (вообще никто)
\end{enumerate}

\subsubsection*{Ярославичи: Изяслав, святослав, Всеволод}

\subsubsection*{Святополк Изяславович}

В его правление было написана "Повесть временных лет"

\subsubsection*{Владимир Всеволодович Мономах}

Мономах - не прозвище, а фамилия.

\subsubsection*{Мстислав Владимирович Великий}

Последний князь древней Руси. Далее - феодальная раздромленность.

