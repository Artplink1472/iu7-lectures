\begin{definition}
  \textbf{Исполняемый файл} - файл, содержащий прогрумму в виде, в котором она может выполнена компьютеров (т.е. в машинном коде).
\end{definition}

Получение исполняемых файлов требует выполнения \textit{компиляции}.

\begin{definition}
  \textbf{Компилятор} - программа для преобразования исходного текста другой программы на определнногом языке в объектный модуль (файл с машинным кодом).
\end{definition}

\section{Язык программирования Python}

\begin{definition}
  \textbf{Python} - высокоуровневый язык программирования общего назначения. Интепретируемый. Является полностью объектно-ориентированным.
\end{definition}

Программа $\to $ Модули $\to $ Операторы $\to $ Выражения $\to $ Лексемы.

\begin{definition}
  Символы алфавита любого языка программирования образуют \texit{лексемы}. По умолчанию - кодировка `UTF-8`.
\end{definition}

\begin{definition}
  \textbf{Лексемы} - минимальная единица языка, имеющая самостоятельный смысл.
\end{definition}

Пять типов лексем:
\begin{enumerate} 
  \item Ключевые слова
  \item Идентификаторы
  \item Литералы
  \item Операции
  \item Знаки пунктуации
\end{enumerate}

\subsubsection*{Строки программы}

\textbf{Физическая строка исходного файла} - строка, заканчивающаяся символом конца строки.

Программы Python разделены на несколько \texit{логических строк}. \textbf{Логическая строка} содержит одну или несколько физических строк, сеодиняющихся правилами языка.

  Ведущие пробельные символы (пробелы и табуляции) в начале строки используются в Python для определения группы инструкций, как единого целого, составной инструмент или блока.

Текст программы должен начинаться с комментария:
\begin{enumerate}
  \item Назначение программы
  \item Автор
\end{enumerate}

Программа должна делиться на три блока:
\begin{enumerate}
  \item Ввод данных
  \item Вычисления
  \item Вывод данных
\end{enumerate}

\subsection{Правила оформления кода на Python}

\begin{itemize}
  \item Программный код должен быть подробно закомментирован
  \item Программа должна выдавать корректный ответ.
  \item При выводе числовых значений мы выводим 5-7 значащих цифр.
  \item При вводе данных должно быть приглашение.
  \item При вводе данных должно быть приглашение.
  \item Если требуется ввод нескольких переменных в одной строке, то необходимо указать, какой разделитель используется.
  \item Исходный код должен оформляться согласно стандарту \texttt{PEP-8} (будут бить).
  \item За транслит тоже будут бить.
  \item Строка должна быть не более 80 символов.
  \item Большие числа выводим в \textit{инженерном виде}. 
\end{itemize}

\subsection{Типы данных}

\begin{definition}
  \textbf{Данные} - полдающееся многократной интепретаций представление инфомрации в формализованном виде, пригодном для передачи, связи или обработки.
\end{definition}

\begin{definition}
  \textbf{Тип данных} - множество значений и операций над этими значениями.
\end{definition}

Основные способы классификации даннхы:
\begin{itemize}
  \item Скалярные и нескалярные
  \item Самостоятельные и зависимые (в том числе ссылочные)
\end{itemize}

Примеры самостоятельных скалярных типов: 
\begin{itemize}
  \item целые
  \item логические
\end{itemize}

Примеры нескалярных типов: массивы, списки, структуры.
\begin{itemize}
  \item массивы
  \item списки
  \item структуры
\end{itemize}

Основные типы данных Python
\begin{itemize}
  \item Числа
  \item Строки
  \item Списки
  \item Словари
  \item Кортежи
  \item Файлы
  \item Множества
  \item Булевы (True/False)
\end{itemize}

\subsubsection{Числовые типы}
\begin{itemize}
  \item Целые
  \item С плавающей запятой
  \item Комплексные
  \item Десятичные с фиксированной точностью
  \item Рациональные дроби
\end{itemize}

\subsubsection{Операции над числами}

Приоритеты:
\begin{enumerate}
  \item Возведение в степень
  \item Умножение, деление, взятие остатка
  \item Сложение, вычитание
  \item Побитовое И
  \item Побитовое исключающее ИЛИ
  \item Побитовое ИЛИ
\end{enumerate}

\subsection{Переменные}

\begin{definition}
  \textbf{Переменная} - проименованная область памяти, которую можно использовать для хранения данных, а её значение можно изменять в ходе выполнения программы.
\end{definition}

\begin{definition}
  \textbf{Имя переменной} - строка символов, которая её идентифицирует (отличает от других переменных и объектов программы).
\end{definition}

Идентификаторы в Python:
\begin{itemize}
  \item Заглавные и строчные символы латиницы
  \item Цифры
  \item Символ нижнего подчёркивания
\end{itemize}

