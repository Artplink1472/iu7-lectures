\section{Координаты вектора. Действия с векторами}

Пусть:
\begin{gather*}
  \vec{a} = \{x_a, y_a, x_a\} \\
  \vec{b} = \{x_b, y_b, z_b \}
\end{gather*}

\begin{note}
  Два вектора равны, если равны соответствующие координаты.
\end{note}

Тогда:
\begin{gather*}
  \vec{c} = \vec{a} + \vec{b} = \{x_a + x_b, y_a + y_b, z_a + z_b\} \\ 
  k \vec{a} = \{k x_a, k y_a, k z_a\} 
\end{gather*}
\begin{note}
  $k \vec{a} = k \cdot \{ \ldots \} $ - так записывать нельзя!
\end{note}

Если $\vec{a} \parallel \vec{b}$, то $\vec{b} = \lambda \vec{a}, где \lambda = const$
\begin{gather*}
  \begin{cases}
    x_b = \lambda x_a \\
    y_b = \lambda y_a \\
    z_b = \lambda z_b
  \end{cases}
  \implies 
  \frac{x_a}{x_b} = \frac{y_a}{y_b} = \frac{z_a}{z_b}
\end{gather*}

\subsubsection*{Расчёт косинуса угла по разложению в базисе}

\begin{eg}
  В $V_2$:
  \begin{gather*}
    \vec{a} = \{x_a, y_a, z_a\} \\
    |\vec{a}| = \sqrt{x_a^2 + y_a^2 + z_a^2} \\
    \cos \alpha = \frac{x_a}{|\vec{a}|} \\
    \cos \beta = \frac{y_a}{|\vec{a}|}
  \end{gather*}
\end{eg}

\begin{eg}
  Для $V_3$: \\
  \begin{gather*}
    \begin{matrix}
      \cos \alpha = \frac{x_a}{|\vec{a}|} \\
      \cos \beta = \frac{y_a}{|\vec{a}|} \\
      \cos \gamma = \frac{z_a}{|\vec{a}|}
    \end{matrix}
    \qquad
    \begin{matrix}
      x_a = |\vec{a}| \cos \alpha \\
      y_a = |\vec{a}| \cos \beta \\
      z_a = |\vec{a}| \cos \gamma \\
    \end{matrix}
  \end{gather*}
  Возведём в квадрат:
  \begin{gather*}
    |\vec{a}|^2 \cos^2 \alpha + |\vec{a}|^2 \cos^2 \beta + |\vec{a}|^2 \cos^2 \gamma = x_a^2 + y_a^2 + z_a^2 = |\vec{a}|^2 \\
    \implies \cos^2 \alpha + \cos^2 \beta + \cos^2 \gamma = 1
  \end{gather*}
  В результате получаем орт вектора $\vec{a}$: \[
  \vec{e_a} = \{\cos \alpha, \cos \beta, \cos \gamma\} 
  \] 
\end{eg}

\subsection{Скалярное произведение векторов}

\begin{definition}
  \textbf{Скалярным произведением} векторов $\vec{a}, \vec{b}$ называется \textit{число}  равное произведению длин этих векторов на косинус угла между ними.\[
  \vec{a} \cdot \vec{b} = |\vec{a}| \cdot |\vec{b}| \cdot \cos \phi
  \] 
\end{definition}

\subsubsection{Свойства скалярного произведения}

\begin{enumerate}
  \item Коммунитативность \[
    \vec{a} \cdot \vec{b} = \vec{b} \cdot \vec{a}
  \] 
  \item
  \begin{gather*}
    \vec{a}^2 \ge 0 \\
    \vec{a}^2 = 0 \iff \vec{a} = \vec{0} \\
    \vec{a}^2 = |\vec{a}|^2
  \end{gather*}
  \item Дистрибутивность \[
      \left( \vec{a} + \vec{b} \right) \cdot \vec{c} = \vec{a} \cdot \vec{c} + \vec{b} \cdot \vec{c}
  \]
  \item Ассоциативность \[
    \left( \lambda \vec{a} \right) \cdot \vec{b} = \lambda \left( \vec{a} \cdot \vec{b} \right) 
  \]
\end{enumerate}

\subsubsection{Формула для вычисления скалярного произведения двух векторов, заданных ортонормированным базисом}

\begin{gather*}
  \vec{a} \cdot \vec{b} = |\vec{a}| \cdot |\vec{b}| \cos \varphi \\
  \vec{a} \cdot \vec{b} > 0 \text{, если } \varphi \in \left( 0; \frac{\pi}{2} \right)  \\
  \vec{a} \cdot \vec{b} < 0 \text{, если } \varphi \in \left( \frac{\pi}{2}; \pi \right) \\
  \vec{a} \cdot \vec{b} = 0 \text{, если } \varphi = \frac{\pi}{2}
\end{gather*}

Пусть в пространстве $V_3$ с заданным ортонормированном базисе $\vec{i}, \vec{j}, \vec{k}$ заданы вектора $\vec{a}, \vec{b}$:
\begin{gather*}
  \vec{a} = x_a \vec{i} + y_a \vec{j} + z_a \vec{k} \\
  \vec{b} = x_b \vec{i} + y_b \vec{j} + z_b \vec{k} \\
\end{gather*}

Тогда:
\begin{gather*}
  \begin{matrix}
    \vec{i}^2 = \vec{i} \cdot \vec{i} = |\vec{i}|^2 = 1 \\
    \vec{j}^2 = \vec{j} \cdot \vec{j} = |\vec{j}|^2 = 1 \\ 
    \vec{k}^2 = \vec{k} \cdot \vec{k} = |\vec{k}|^2 = 1 \\
  \end{matrix}
  \qquad 
  \begin{matrix}
    \vec{i} \perp \vec{j} \implies \vec{i} \cdot \vec{j} = 0 \\ 
    \vec{i} \perp \vec{k} \implies \vec{i} \cdot \vec{k} = 0 \\ 
    \vec{j} \perp \vec{k} \implies \vec{j} \cdot \vec{k} = 0 \\ 
  \end{matrix}
\end{gather*}

\begin{gather*}
  \vec{a} \cdot \vec{b}
  = \left( x_a \vec{i} + y_a \vec{j} + z_a \vec{k} \right) \left( x_b \vec{i} + y_b \vec{j} + z_b \vec{k} \right) \\
  = x_a x_b \vec{i}^2 + x_a y_b (\vec{i} \cdot \vec{j}) + x_a z_b (\vec{i} \cdot \vec{k}) \\
  + y_a x_a (\vec{i} \cdot \vec{j}) + y_a y_b \vec{j}^2 + y_a z_b (\vec{j} \cdot \vec{k}) \\
  + z_a x_b (\vec{i} \cdot \vec{k}) + z_a y_b (\vec{j} \cdot \vec{k}) + z_a z_b \vec{k}^2 \\
  = x_a x_b + y_a y_b + z_a z_b \\
\end{gather*}

\begin{center}
  \boxed{\vec{a} \cdot \vec{b} = x_a x_b + y_a y_b + z_a z_b}
\end{center}


\subsubsection{Формула косинуса между векторами, заданными ортонормированным базисом}

Т.к. $\vec{a} \vec{b} = |\vec{a}| |\vec{b}| \cos \phi$, то:
\begin{gather*}
  \cos \varphi = \frac{\vec{a} \vec{b}}{|\vec{a}| |\vec{b}|} \\
  = \frac{x_a x_b + y_a y_b + z_a z_b}{|\vec{a}| \cdot |\vec{b}|} \\
  = \frac{x_a x_b + y_a y_b + z_a z_b}{\sqrt{x_a^2 + y_a^2 + z_a^2} +\sqrt{x_b^2 + y_b^2 + z)b^2} } \\
  \boxed{\cos \varphi = \frac{x_a x_b + y_a y_b + z_a z_b}{\sqrt{x_a^2 + y_a^2 + z_a^2} +\sqrt{x_b^2 + y_b^2 + z)b^2}}}
\end{gather*}


\subsection{Векторное произведение векторов}

\begin{definition}
  Тройка векторов называется \textbf{правой}, если кратчайший поворот от вектора $\vec{a}$ к $\vec{b}$ осуществляется \textit{против часовой стрелки} (смотря из конца вектора $\vec{c}$).
\end{definition}

\begin{definition}
  Тройка векторов называется \textbf{левой}, если кратчайший поворот от вектора $\vec{a}$ к $\vec{b}$ осуществляется \textit{по часовой стрелки} (смотря из конца вектора $\vec{c}$).
\end{definition}

\begin{definition}
  \textbf{Векторным произведением} векторов $\vec{a}$ и $\vec{b}$ называется вектор $\vec{c}$, который удовлетворяет следующему условию:
  \begin{enumerate}
    \item $\vec{c}$ ортогонален векторам $\vec{a}$ и $\vec{b}$ (перпендикулярен плоскости, в которой лежат вектора $\vec{a}$ и $\vec{b}$);
    \item $\vec{c} = |\vec{a}| |\vec{b}| \cdot \sin \phi$
    \item Вектора $\vec{a}, \vec{b}, \vec{c}$ образуют \textit{правую} тройку векторов.
  \end{enumerate}
  Обозначение: \[
    \vec{a} \times \vec{b} \text{ или } [\vec{a}, \vec{b}]
  \] 
\end{definition}

\subsubsection{Свойства векторного произведения векторов}

\begin{enumerate}
  \item Антикомунитативность \[
    \vec{a} \times \vec{b} = - \vec{b} \times \vec{a}
  \] 
\item Дистрибутивность \[
    \left( \vec{a_1} + \vec{a_2} \right) \times \vec{b} = \vec{a_1} \times \vec{b} + \vec{a_2} \times  \vec{b} 
  \]  
  \item Ассоциативность \[
    \left( \lambda \vec{a} \right) \times \vec{b} = \lambda \left( \vec{a} \times \vec{b} \right)  
  \]
\end{enumerate}

\subsubsection{Геометрическое приложение векторов.}
Пусть $\vec{a} = \{x_a y_a, x_a\}$ и $\vec{b} = \{x_b, y_b, z_b\}$. Совместим начала этих векторов и достроим до параллелограмма. Тогда площадь этого параллелограмма будет равна модулю векторного произведения этих векторов.
% Рисунок параллелограмчика

\begin{eg}
  \begin{gather*}
    A(1, 2, -1), \quad B(-1, 1, 0), \quad C(0, -1, 2) \\
    \overrightarrow{AB} = \{-2, -1, 1\} \\
    \overrightarrow{AC} = \{-1, -3, 3\} \\
    \\
    \overrightarrow{AB} \times \overrightarrow{AC} = 
    \begin{vmatrix}
      \vec{i} & \vec{j} & \vec{k} \\
      -2 & -1 & 1 \\
      -1 & -3 & 3
    \end{vmatrix} = \\
    \vec{i} \cdot (-1)^{1+1}
    \begin{vmatrix}
      -1 & 1 \\
      -3 & 3
    \end{vmatrix} +
    \vec{j} \cdot (-1)^{1+2} 
    \begin{vmatrix}
      -2 & 1 \\
      -1 & 3
    \end{vmatrix} +
    \vec{k} \cdot (-1)^{1+3}
    \begin{vmatrix}
      -2 & -1 \\
      -1 & -3
    \end{vmatrix} =\\
    0 \vec{i} + 5 \vec{j} + 5 \vec{k}
  \end{gather*}
  \begin{gather*}
    \vec{c} = \{0, 5, 5\} \implies |\vec{c}| = \sqrt{50} = 5\sqrt{2} \\
    S_{ABC} = \frac{1}{2} S_{ABCD} = \frac{1}{2} \cdot 5\sqrt{2} = \frac{5}{\sqrt{2}} 
  \end{gather*} 
\end{eg}

\subsection{Смешанное произведение}

\begin{definition}
  Смешанное поизведение векторов $\vec{a}, \vec{b}, \vec{c}$ называется скалярное произведения первых двух векторов $\vec{a}$ и $\vec{b}$ на третий вектор $\vec{c}$.
  \[
  \vec{a} \vec{b} \vec{c} = (\vec{a} \cdot \vec{b}) \times \vec{c}
  \] 
\end{definition}

\subsubsection{Свойства смешанных произведений}

\begin{enumerate}
  \item \textbf{Свойство перестановки (кососимметричности)} \\
    \begin{gather*}
      \vec{a} \vec{b} \vec{c} = \vec{c} \vec{a} \vec{b} = \vec{b} \vec{c} \vec{a} = -\vec{b} \vec{a} \vec{c} = - \vec{c} \vec{b} \vec{a} = - \vec{a} \vec{c} \vec{b}
    \end{gather*}

  \item Три вектора компланарны тогда и только тогда, когда их смешанное произведение равно 0. \[
  \vec{a}, \vec{b}, \vec{c} \text{ - компланарны} \iff \vec{a} \vec{b} \vec{c} = 0
  \] 
    \begin{note}
      $\vec{a} \vec{b} \vec{c} > 0$, если $\vec{a}, \vec{b}, \vec{c}$ - правая тройка векторов. \\
      $\vec{a} \vec{b} \vec{c} < 0$, если $\vec{a}, \vec{b}, \vec{c}$ - левая тройка векторов.
      % Чертёж
    \end{note}

  \item \textbf{Свойство ассоциативности} \\
    \begin{gather*}
      (\lambda \vec{a}) \vec{b} \vec{c} = \lambda (\vec{a} \vec{b} \vec{c})
    \end{gather*}
      \begin{proof}
        \[(\lambda \vec{a}) \vec{b} \vec{c} = (\lambda \vec{a}) \vec{d} = \lambda (\vec{a} \vec{d}) = \lambda (\vec{a} (\vec{b} \vec{c})) = \lambda (\vec{a} \vec{b} \vec{c})\]
      \end{proof}
      \begin{note}
        Примечание: это работает для любого положения $\lambda$.
      \end{note}

  \item \textbf{Свойство коммутативности} 
    \[
      (\vec{a_1} + \vec{a_2}) \vec{b} \vec{c} = \vec{a_1} \vec{b} \vec{c} + \vec{a_2} \vec{b} \vec{c}
    \] 
    \begin{proof}
      \begin{align*}
        (\vec{a_1} + \vec{a_2}) \vec{b} \vec{c} 
          &= (\vec{a_1} + \vec{a_2}) \vec{d}\\
          &= \vec{a_1} \vec{d} + \vec{a_2} \vec{d}\\
          &= \vec{a_1} (\vec{b} \vec{c}) + \vec{a_2} (\vec{b} \vec{c})\\
          &= \vec{a_1} \vec{b} \vec{c} + \vec{a_2} \vec{b} \vec{c}
      \end{align*}
    \end{proof}
    \begin{note}
      Работает не только для $\vec{a}$, но и векторов $\vec{b}$ и $\vec{c}$.
    \end{note}
\end{enumerate}

\subsubsection{Формула смешанного произведения трёх векторов в правом ортонормированном базисе}

Пусть $\vec{a}, \vec{b}, \vec{c}$ заданы координатами:
\begin{gather*}
  \vec{a} = \{x_a, y_a, x_a\} \\
  \vec{b} = \{x_b, y_b,. z_b\} \\
  \vec{c} = \{x_c, y_c, z_c\} 
\end{gather*}

Найдём смешанное произведение:
\begin{gather*}
  \vec{a} \vec{b} \vec{c} = (\vec{a} \times \vec{b}) \cdot \vec{c} = \{
    \begin{vmatrix}
      y_a & z_a \\
      y_b & z_b 
    \end{vmatrix},
    - \begin{vmatrix}
      x_a & z_a \\
      x_b & z_b 
    \end{vmatrix},
    \begin{vmatrix}
      x_a & y_a \\
      x_b & y_b
    \end{vmatrix}
  \} \cdot \vec{c} = \\
    = \begin{vmatrix}
      y_a & z_a \\
      y_b & z_b 
    \end{vmatrix} \cdot x_c
    - \begin{vmatrix}
      x_a & z_a \\
      x_b & z_b 
    \end{vmatrix} \cdot y_c
    + \begin{vmatrix}
      x_a & y_a \\
      x_b & y_b
    \end{vmatrix} \cdot z_c = \\
    =
    \begin{vmatrix}
      x_a & y_a & z_a \\
      x_b & y_b & z_b \\
      x_c & y_c & z_c 
    \end{vmatrix}
\end{gather*}

Т.е. \[
\vec{a} \vec{b} \vec{c} =
\begin{vmatrix}
  x_a & y_a & z_c \\
  x_b & y_b & z_b \\
  x_c & y_c & z_c \\
\end{vmatrix}
\]
\subsubsection{Геометрическое приложение смешанного произведения}

Пусть $\vec{a}, \vec{b}, \vec{c}$. Совместим начала этих векторов и достроим до параллелипипеда. Тогда $V_{paral} = |\vec{a} \vec{b} \vec{c}|$.
% Рисуночек 
\begin{note}
  \[
    V_{pyramid} = \frac{1}{6} V_{paral} = \frac{1}{6} \cdot |\vec{a} \vec{b} \vec{c}| 
  \] 
\end{note}

