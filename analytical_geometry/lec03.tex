\begin{theorem}
  \textit{О разложении вектора по базису} \\
  Любой вектор можно разложить по базису и при этом единственным образом.
\end{theorem}
\begin{proof}
  Пусть в пространстве $V_3$ зафиксирован базис $\vec{e_1}, \vec{e_2}, \vec{e_3}$. Возьмём вектор $\vec{x}$. Тогда система векторов $\vec{x}, \vec{e_1}, \vec{e_2}, \vec{e_3}$ - линейно зависима, если вектор $\vec{x}$ можно представить в виде линейной комбинации векторов $\vec{e_1}, \vec{e_2}, \vec{e_3}$: \[
  \vec{x} = \lambda_1 \vec{e_1} + \lambda_2 \vec{e_2} + \lambda_3 + \vec{e_3} \tag{1}
\]
  Предположим, что разложение вектора $\vec{x}$ - не единственное. \[
  \vec{x} = \rho \vec{e_1} + \rho \vec{e_2} + \rho \vec{e_3} \tag{2}
\]
  Вычтем из (1) уранвение (2). Тогда: \[
    \vec{0} = \left( \lambda_1 - \rho_1 \right) \vec{e_1} + \left( \lambda_2 - \rho_2 \right) \vec{e_2} + \left( \lambda_3 - \rho_3 \right) \vec{e_3} \tag{3}
\]
Поскольку базисные вектора $\vec{e_1}, \vec{e_2}, \vec{e_3}$ - линейно независимы, то выражение (3) представляет собой тривиальную линейную комбинацию векторов $\vec{e_1}, \vec{e_2}, \vec{e_3}$, равную нулю. Тогда получаем:
  \begin{gather*}
    \begin{matrix}  
      \lambda_1 - \delta_1 = 0 \\
      \lambda_2 - \delta_2 = 0 \\
      \lambda_3 - \delta_3 = 0 \\
    \end{matrix}
    \quad \implies \quad
    \begin{matrix}
      \lambda_1 = \delta_1 \\
      \lambda_2 = \delta_2 \\
      \lambda_3 = \delta_3 \\
    \end{matrix}
  \end{gather*}
  Коэффициенты равны, что и требовалось доказать.
\end{proof}
\begin{eg}
  Пусть в пространстве $V_2$ зафиксирован базис $\vec{i}, \vec{j}$.
  \begin{gather*}
    |\vec{i}| = 1, \quad |\vec{j}| = 1 \\
    \vec{a} = \overrightarrow{OA} + \overrightarrow{OB} \\
    \overrightarrow{OA} \parallel \vec{i} \implies \overrightarrow{OA} = x_a \vec{i} \\
    \overrightarrow{OB} \parallel \vec{j} \implies \overrightarrow{OB} = y_a \vec{j} \\
    \implies \vec{a} = x_a \vec{i} + y_a \vec{j}
  \end{gather*}
\end{eg}
\begin{eg}
  Пусть в пространстве $V_3$ зафиксирован ортонормированный базис $\vec{i}, \vec{j}, \vec{k}$
  Тогда:
  \begin{gather*}
    \vec{a} = \{x_a, y_a, z_a\} \\
    \vec{a} = x_a \vec{i} + y_a \vec{j} + z_a \vec{k}
  \end{gather*}
\end{eg}

\exercise{}
Разложить $\vec{a}$ по векторам $\vec{a}, \vec{b}, \vec{c}$. \\
Дано:
\begin{gather*}
  \vec{a} = 3 \vec{i} - 4 \vec{j} \\
  \vec{b} = 2 \vec{i} -   \vec{j} \\
  \vec{c} = - \vec{i} - 5 \vec{j}
\end{gather*}
Решение:
\begin{gather*}
  \vec{a} = \alpha \vec{b} + \beta \vec{c} \\
  3 \vec{i} - 4 \vec{j} = \alpha(2 \vec{i} + \vec{j}) + \beta (- \vec{i} + 5 \vec{j}) \\
  3 \vec{i} - 4 \vec{j} = (2 \alpha - \beta) \vec{i} + (\alpha + 5 \beta) \vec{j} \implies \\
  \begin{cases}
    3 = 2 \alpha - \beta \\
    -4 = \alpha + 5 \beta
  \end{cases} 
  \implies 
  \begin{cases}
    \beta = -1 \\
    \alpha - 1
  \end{cases}
\end{gather*}

\begin{note}
  Два вектора равны, если равны соответствующие координаты.
\end{note}

\subsection{Координаты вектора. Действия с векторами}

Пусть:
\begin{gather*}
  \vec{a} = \{x_a, y_a, x_a\} \\
  \vec{b} = \{x_b, y_b, z_b \}
\end{gather*}

Тогда:
\begin{gather*}
  \vec{c} = \vec{a} + \vec{b} = \{x_a + x_b, y_a + y_b, z_a + z_b\} \\ 
  k \vec{a} = \{k x_a, k y_a, k z_a\} 
\end{gather*}
\begin{note}
  $k \vec{a} = k \cdot \{ \ldots \} $ - так записывать нельзя!
\end{note}

Если $\vec{a} \parallel \vec{b}$, то $\vec{b} = \lambda \vec{a}, где \lambda = const$
\begin{gather*}
  \begin{cases}
    x_b = \lambda x_a \\
    y_b = \lambda y_a \\
    z_b = \lambda z_b
  \end{cases}
  \implies 
  \frac{x_a}{x_b} = \frac{y_a}{y_b} = \frac{z_a}{z_b}
\end{gather*}

\subsubsection*{Расчёт косинуса угла по разложению в базисе}

\begin{eg}
  В $V_2$:
  \begin{gather*}
    \vec{a} = \{x_a, y_a, z_a\} \\
    |\vec{a}| = \sqrt{x_a^2 + y_a^2 + z_a^2} \\
    \cos \alpha = \frac{x_a}{|\vec{a}|} \\
    \cos \beta = \frac{y_a}{|\vec{a}|}
  \end{gather*}
\end{eg}

\begin{eg}
  Для $V_3$: \\
  \begin{gather*}
    \begin{matrix}
      \cos \alpha = \frac{x_a}{|\vec{a}|} \\
      \cos \beta = \frac{y_a}{|\vec{a}|} \\
      \cos \gamma = \frac{z_a}{|\vec{a}|}
    \end{matrix}
    \qquad
    \begin{matrix}
      x_a = |\vec{a}| \cos \alpha \\
      y_a = |\vec{a}| \cos \beta \\
      z_a = |\vec{a}| \cos \gamma \\
    \end{matrix}
  \end{gather*}
  Возведём в квадрат:
  \begin{gather*}
    |\vec{a}|^2 \cos^2 \alpha + |\vec{a}|^2 \cos^2 \beta + |\vec{a}|^2 \cos^2 \gamma = x_a^2 + y_a^2 + z_a^2 = |\vec{a}|^2 \\
    \implies \cos^2 \alpha + \cos^2 \beta + \cos^2 \gamma = 1
  \end{gather*}
  В результате получаем орт вектора $\vec{a}$: \[
  \vec{e_a} = \{\cos \alpha, \cos \beta, \cos \gamma\} 
  \] 
\end{eg}

\subsection{Скалярное произведение векторов}

\begin{definition}
  \textbf{Скалярным произведением} векторов $\vec{a}, \vec{b}$ называется \textit{число}  равное произведению длин этих векторов на косинус угла между ними.\[
  \vec{a} \cdot \vec{b} = |\vec{a}| \cdot |\vec{b}| \cdot \cos \phi
  \] 
\end{definition}

\subsubsection{Свойства скалярного произведения}

\begin{enumerate}
  \item Коммунитативность \[
    \vec{a} \cdot \vec{b} = \vec{b} \cdot \vec{a}
  \] 
  \item
  \begin{gather*}
    \vec{a}^2 \ge 0 \\
    \vec{a}^2 = 0 \iff \vec{a} = \vec{0} \\
    \vec{a}^2 = |\vec{a}|^2
  \end{gather*}
  \item Дистрибутивность \[
      \left( \vec{a} + \vec{b} \right) \cdot \vec{c} = \vec{a} \cdot \vec{c} + \vec{b} \cdot \vec{c}
  \]
  \item Ассоциативность \[
    \left( \lambda \vec{a} \right) \cdot \vec{b} = \lambda \left( \vec{a} \cdot \vec{b} \right) 
  \]
\end{enumerate}

\subsubsection{Формула для вычисления скалярного произведения двух векторов, заданных ортонормированным базисом}

\begin{gather*}
  \vec{a} \cdot \vec{b} = |\vec{a}| \cdot |\vec{b}| \cos \varphi \\
  \vec{a} \cdot \vec{b} > 0 \text{, если } \varphi \in \left( 0; \frac{\pi}{2} \right)  \\
  \vec{a} \cdot \vec{b} < 0 \text{, если } \varphi \in \left( \frac{\pi}{2}; \pi \right) \\
  \vec{a} \cdot \vec{b} = 0 \text{, если } \varphi = \frac{\pi}{2}
\end{gather*}

Пусть в пространстве $V_3$ с заданным ортонормированном базисе $\vec{i}, \vec{j}, \vec{k}$ заданы вектора $\vec{a}, \vec{b}$:
\begin{gather*}
  \vec{a} = x_a \vec{i} + y_a \vec{j} + z_a \vec{k} \\
  \vec{b} = x_b \vec{i} + y_b \vec{j} + z_b \vec{k} \\
\end{gather*}

Тогда:
\begin{gather*}
  \begin{matrix}
    \vec{i}^2 = \vec{i} \cdot \vec{i} = |\vec{i}|^2 = 1 \\
    \vec{j}^2 = \vec{j} \cdot \vec{j} = |\vec{j}|^2 = 1 \\ 
    \vec{k}^2 = \vec{k} \cdot \vec{k} = |\vec{k}|^2 = 1 \\
  \end{matrix}
  \qquad 
  \begin{matrix}
    \vec{i} \perp \vec{j} \implies \vec{i} \cdot \vec{j} = 0 \\ 
    \vec{i} \perp \vec{k} \implies \vec{i} \cdot \vec{k} = 0 \\ 
    \vec{j} \perp \vec{k} \implies \vec{j} \cdot \vec{k} = 0 \\ 
  \end{matrix}
\end{gather*}

\begin{gather*}
  \vec{a} \cdot \vec{b}
  = \left( x_a \vec{i} + y_a \vec{j} + z_a \vec{k} \right) \left( x_b \vec{i} + y_b \vec{j} + z_b \vec{k} \right) \\
  = x_a x_b \vec{i}^2 + x_a y_b (\vec{i} \cdot \vec{j}) + x_a z_b (\vec{i} \cdot \vec{k}) \\
  + y_a x_a (\vec{i} \cdot \vec{j}) + y_a y_b \vec{j}^2 + y_a z_b (\vec{j} \cdot \vec{k}) \\
  + z_a x_b (\vec{i} \cdot \vec{k}) + z_a y_b (\vec{j} \cdot \vec{k}) + z_a z_b \vec{k}^2 \\
  = x_a x_b + y_a y_b + z_a z_b \\
\end{gather*}

\begin{center}
  \boxed{\vec{a} \cdot \vec{b} = x_a x_b + y_a y_b + z_a z_b}
\end{center}



\subsubsection{Формула косинуса между векторами, заданными ортонормированным базисом}

Т.к. $\vec{a} \vec{b} = |\vec{a}| |\vec{b}| \cos \phi$, то:
\begin{gather*}
  \cos \varphi = \frac{\vec{a} \vec{b}}{|\vec{a}| |\vec{b}|} \\
  = \frac{x_a x_b + y_a y_b + z_a z_b}{|\vec{a}| \cdot |\vec{b}|} \\
  = \frac{x_a x_b + y_a y_b + z_a z_b}{\sqrt{x_a^2 + y_a^2 + z_a^2} +\sqrt{x_b^2 + y_b^2 + z)b^2} } \\
  \boxed{\cos \varphi = \frac{x_a x_b + y_a y_b + z_a z_b}{\sqrt{x_a^2 + y_a^2 + z_a^2} +\sqrt{x_b^2 + y_b^2 + z)b^2}}}
\end{gather*}


\subsection{Векторное произведение векторов}

\begin{definition}
  Тройка векторов называется \textbf{правой}, если кратчайший поворот от вектора $\vec{a}$ к $\vec{b}$ осуществляется \textit{против часовой стрелки} (смотря из конца вектора $\vec{c}$).
\end{definition}

\begin{definition}
  Тройка векторов называется \textbf{левой}, если кратчайший поворот от вектора $\vec{a}$ к $\vec{b}$ осуществляется \textit{по часовой стрелки} (смотря из конца вектора $\vec{c}$).
\end{definition}

\begin{definition}
  \textbf{Векторным произведением} векторов $\vec{a}$ и $\vec{b}$ называется вектор $\vec{c}$, который удовлетворяет следующему условию:
  \begin{enumerate}
    \item $\vec{c}$ ортогонален векторам $\vec{a}$ и $\vec{b}$ (перпендикулярен плоскости, в которой лежат вектора $\vec{a}$ и $\vec{b}$);
    \item $\vec{c} = |\vec{a}| |\vec{b}| \cdot \sin \phi$
    \item Вектора $\vec{a}, \vec{b}, \vec{c}$ образуют \textit{правую} тройку векторов.
  \end{enumerate}
  Обозначение: \[
    \vec{a} \times \vec{b} \text{ или } [\vec{a}, \vec{b}]
  \] 
\end{definition}

\subsubsection{Свойства векторного произведения векторов}

\begin{enumerate}
  \item Антикомунитативность \[
    \vec{a} \times \vec{b} = - \vec{b} \times \vec{a}
  \] 
\item Дистрибутивность \[
    \left( \vec{a_1} + \vec{a_2} \right) \times \vec{b} = \vec{a_1} \times \vec{b} + \vec{a_2} \times  \vec{b} 
  \]  
  \item Ассоциативность \[
    \left( \lambda \vec{a} \right) \times \vec{b} = \lambda \left( \vec{a} \times \vec{b} \right)  
  \]
\end{enumerate}

