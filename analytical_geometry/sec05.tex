\section{Прямая на плоскости}

\subsection{Каноническое уравнение}

Пусть прямая $l$ проходит через точку $M_0(x_0, y_0)$ и задана направляющим вектором $\vec{S} = \{m, n\}$ (т.е. вектор паралеллен прямой). Выберем на прямой $l$ произвольную точку $M$.
Составим $\overrightarrow{M_0M} = \{x - x_0, y - y_0, z - z_0\}$. 
\[
  \overrightarrow{M_0M} \parallel \vec{s} \implies \boxed{\frac{x - x_0}{m} = \frac{y - y_0}{n}}
\]

\subsection{Параметрическое уравнение}

Пусть прямая $l$ задана каноническим уравнением: \[
\frac{x - x_0}{m} = \frac{y - y_0}{n}
\] 
Обозначим коеффициент пропорциональности через $t$. Тогда:
\begin{gather*}
  \begin{matrix}
    \frac{x - x_0}{m} = t \\
    \frac{y - y_0}{n} = t
  \end{matrix} \implies \boxed{\begin{cases}
    x = x_0 + mt \\
    y = y_0 + nt
  \end{cases}}
\end{gather*}
 
\subsection{Через две точки}

Пусть прямая $l$ проходит через точки $M_0(x_0, y_0)$ и $M(x, y)$. Выберем на прямой $l$ произвольную точку $M_1(x_1, y_1)$. Составим два вектора $\overrightarrow{M_0M}$, $\overrightarrow{M_0, M_1}$.
\begin{gather*}
  \overrightarrow{M_0M} = \{x - x_0, y - y_0\} \\
  \overrightarrow{M_0M_1} = \{x_1 - x_0, y_1 - y_0\}
\end{gather*}

Т.к. вектора коллинеарны, то и соответствующие координаты пропорциональны:
 \[
  \boxed{\frac{x - x_0}{x_1 - x_0} = \frac{y - y_0}{y_1 - y_0}}
\]

\subsection{В отрезках}

% Рисуночек
Пусть прямая $l$ отсекает от координатного угла отрезки $a$ и $b$. Тогда прямая $l$ проходит через точки $A(0, a)$ и  $B(b, 0)$.
\begin{gather*}
  \frac{x - a}{0 - a} = \frac{y - 0}{b - 0} \implies \boxed{\frac{x}{a} + \frac{y}{b} = 1}
\end{gather*}


