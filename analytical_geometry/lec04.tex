\subsubsection{Формула для вычисления векторного произведения в правом ортонормированном базисе}

Пусть $V_3$ определен правый ортонормированный базис $\vec{i}, \vec{j}, \vec{k}$. Рассмотрим векторное произведение:
\begin{align*}
  \vec{i} \times \vec{i} = \vec{0} \\
  \vec{j} \times \vec{j} = \vec{0} \\
  \vec{k} \times \vec{k} = \vec{0} \\
\end{align*}
\begin{align*}
  \vec{i} \times \vec{j} &=  \vec{k} \\
  \vec{i} \times \vec{k} &= -\vec{j} \\
  \vec{j} \times \vec{k} &=  \vec{i} \\
  \vec{j} \times \vec{i} &= -\vec{k} \\
  \vec{k} \times \vec{i} &=  \vec{j} \\
  \vec{k} \times \vec{j} &= -\vec{i} \\
\end{align*}

Рассмотрим два вектора $\vec{a} = \{x_a, y_a, z_a\}$ и $\vec{b} = \{x_b, y_b, z_b\}$. Тогда можно записать разложение этих векторов по базису:
\begin{align*}
  \vec{a} &= x_a \vec{i} + y_a \vec{j} + z_a \vec{k} \\
  \vec{b} &= x_b \vec{i} + y_b \vec{j} + z_b \vec{k}
\end{align*}

Найдем векторное произведение этих векторов:
\begin{gather*}
  \begin{split}
    \vec{a} \times \vec{b} &= (x_a \vec{i} + y_a \vec{j} + z_a \vec{k}) \times (x_b \vec{i} + y_b \vec{j} + z_b \vec{k}) =\\
     &= x_a x_b (\vec{i} \times \vec{i}) + x_a y_b (\vec{i} \times \vec{j}) + x_a z_b (\vec{i} \times \vec{k}) +\\
     &+ y_a x_a (\vec{j} \times \vec{i}) + y_a y_b (\vec{j} \times \vec{j}) + y_a z_b (\vec{j} \times \vec{k}) +\\
     &+ z_a x_b (\vec{k} \times \vec{i}) + z_a y_b (\vec{k} \times \vec{j}) + z_a z_b (\vec{k} \times \vec{k}) =\\
     &= x_a y_b \vec{k} - x_a z_b \vec{j} - y_a x_b \vec{k} + y_a z_b \vec{i} + z_a x_b \vec{j} - z_a y_b \vec{j} =\\
     &= (y_a z_b - z_a y_b) \vec{i} - (x_a z_b - z_a x_b) \vec{j} + (x_a y_b - y_a x_b) \vec{k} = \left( * \right) 
  \end{split}
\end{gather*} 

Можно заметить, что это равняется определителю:
\begin{gather*}
  \begin{vmatrix}
    \vec{i} & \vec{j} & \vec{k} \\
    x_a & y_a & z_a \\
    x_b & y_b & z_b 
  \end{vmatrix} = 
  \vec{i} \cdot
  \begin{vmatrix}
    y_a & z_a \\
    y_b & z_b 
  \end{vmatrix}
  - \vec{j} \cdot
  \begin{vmatrix}
    x_a & z_a \\
    x_b & z_b
  \end{vmatrix}
  + \vec{k} \cdot 
  \begin{vmatrix}
    x_a & y_a \\
    x_b & y_b
  \end{vmatrix}
\end{gather*}

Тем самым мы получаем:
\begin{gather*}
  \boxed{
    \vec{a} \times \vec{b} = 
    \left\{ \begin{vmatrix}
        y_a & z_a \\
        y_b & z_b 
    \end{vmatrix},
    - \begin{vmatrix}
      x_a & z_a \\
      x_b & z_b 
    \end{vmatrix},
    \begin{vmatrix}
      x_a & y_a \\
      x_b & y_b
    \end{vmatrix}
  \right\}
}
\end{gather*}

\begin{theorem}
  Для того, чтобы векторы были коллинеарны, необходимо и достаточно, чтобы их векторное произведение было равно нулевому вектору.
\end{theorem}

\subsubsection{Геометрическое приложение векторов.}
Пусть $\vec{a} = \{x_a y_a, x_a\}$ и $\vec{b} = \{x_b, y_b, z_b\}$. Совместим начала этих векторов и достроим до параллелограмма. Тогда площадь этого параллелограмма будет равна модулю векторного произведения этих векторов.
% Рисунок параллелограмчика

\begin{eg}
  \begin{gather*}
    A(1, 2, -1), \quad B(-1, 1, 0), \quad C(0, -1, 2) \\
    \overrightarrow{AB} = \{-2, -1, 1\} \\
    \overrightarrow{AC} = \{-1, -3, 3\} \\
    \\
    \overrightarrow{AB} \times \overrightarrow{AC} = 
    \begin{vmatrix}
      \vec{i} & \vec{j} & \vec{k} \\
      -2 & -1 & 1 \\
      -1 & -3 & 3
    \end{vmatrix} = \\
    \vec{i} \cdot (-1)^{1+1}
    \begin{vmatrix}
      -1 & 1 \\
      -3 & 3
    \end{vmatrix} +
    \vec{j} \cdot (-1)^{1+2} 
    \begin{vmatrix}
      -2 & 1 \\
      -1 & 3
    \end{vmatrix} +
    \vec{k} \cdot (-1)^{1+3}
    \begin{vmatrix}
      -2 & -1 \\
      -1 & -3
    \end{vmatrix} =\\
    0 \vec{i} + 5 \vec{j} + 5 \vec{k}
  \end{gather*}
  \begin{gather*}
    \vec{c} = \{0, 5, 5\} \implies |\vec{c}| = \sqrt{50} = 5\sqrt{2} \\
    S_{ABC} = \frac{1}{2} S_{ABCD} = \frac{1}{2} \cdot 5\sqrt{2} = \frac{5}{\sqrt{2}} 
  \end{gather*} 
\end{eg}

\subsection{Смешанное произведение}

\begin{definition}
  Смешанное поизведение векторов $\vec{a}, \vec{b}, \vec{c}$ называется скалярное произведения первых двух векторов $\vec{a}$ и $\vec{b}$ на третий вектор $\vec{c}$.
  \[
  \vec{a} \vec{b} \vec{c} = (\vec{a} \cdot \vec{b}) \times \vec{c}
  \] 
\end{definition}

\subsubsection{Свойства смешанных произведений}

\begin{enumerate}
  \item \textbf{Свойство перестановки (кососимметричности)} \\
    \begin{gather*}
      \vec{a} \vec{b} \vec{c} = \vec{c} \vec{a} \vec{b} = \vec{b} \vec{c} \vec{a} = -\vec{b} \vec{a} \vec{c} = - \vec{c} \vec{b} \vec{a} = - \vec{a} \vec{c} \vec{b}
    \end{gather*}

  \item Три вектора компланарны тогда и только тогда, когда их смешанное произведение равно 0. \[
  \vec{a}, \vec{b}, \vec{c} \text{ - компланарны} \iff \vec{a} \vec{b} \vec{c} = 0
  \] 
    \begin{note}
      $\vec{a} \vec{b} \vec{c} > 0$, если $\vec{a}, \vec{b}, \vec{c}$ - правая тройка векторов. \\
      $\vec{a} \vec{b} \vec{c} < 0$, если $\vec{a}, \vec{b}, \vec{c}$ - левая тройка векторов.
      % Чертёж
    \end{note}

  \item \textbf{Свойство ассоциативности} \\
    \begin{gather*}
      (\lambda \vec{a}) \vec{b} \vec{c} = \lambda (\vec{a} \vec{b} \vec{c})
    \end{gather*}
      Доказательство.
      \begin{proof*}
        \[(\lambda \vec{a}) \vec{b} \vec{c} = (\lambda \vec{a}) \vec{d} = \lambda (\vec{a} \vec{d}) = \lambda (\vec{a} (\vec{b} \vec{c})) = \lambda (\vec{a} \vec{b} \vec{c})\]
      \end{proof*}
      \begin{note}
        Примечание: это работает для любого положения $\lambda$.
      \end{note}

  \item \textbf{Свойство коммутативности} 
    \[
      (\vec{a_1} + \vec{a_2}) \vec{b} \vec{c} = \vec{a_1} \vec{b} \vec{c} + \vec{a_2} \vec{b} \vec{c}
    \] 
    \begin{proof*}
      \begin{align*}
        (\vec{a_1} + \vec{a_2}) \vec{b} \vec{c} 
          &= (\vec{a_1} + \vec{a_2}) \vec{d}\\
          &= \vec{a_1} \vec{d} + \vec{a_2} \vec{d}\\
          &= \vec{a_1} (\vec{b} \vec{c}) + \vec{a_2} (\vec{b} \vec{c})\\
          &= \vec{a_1} \vec{b} \vec{c} + \vec{a_2} \vec{b} \vec{c}
      \end{align*}
    \end{proof*}
    \begin{note}
      Работает не только для $\vec{a}$, но и векторов $\vec{b}$ и $\vec{c}$.
    \end{note}
\end{enumerate}

\subsubsection{Формула смешанного произведения трёх векторов в правом ортонормированном базисе}

Пусть $\vec{a}, \vec{b}, \vec{c}$ заданы координатами:
\begin{gather*}
  \vec{a} = \{x_a, y_a, x_a\} \\
  \vec{b} = \{x_b, y_b,. z_b\} \\
  \vec{c} = \{x_c, y_c, z_c\} 
\end{gather*}

Найдём смешанное произведение:
\begin{gather*}
  \vec{a} \vec{b} \vec{c} = (\vec{a} \times \vec{b}) \cdot \vec{c} = \{
    \begin{vmatrix}
      y_a & z_a \\
      y_b & z_b 
    \end{vmatrix},
    - \begin{vmatrix}
      x_a & z_a \\
      x_b & z_b 
    \end{vmatrix},
    \begin{vmatrix}
      x_a & y_a \\
      x_b & y_b
    \end{vmatrix}
  \} \cdot \vec{c} = \\
    = \begin{vmatrix}
      y_a & z_a \\
      y_b & z_b 
    \end{vmatrix} \cdot x_c
    - \begin{vmatrix}
      x_a & z_a \\
      x_b & z_b 
    \end{vmatrix} \cdot y_c
    + \begin{vmatrix}
      x_a & y_a \\
      x_b & y_b
    \end{vmatrix} \cdot z_c = \\
    =
    \begin{vmatrix}
      x_a & y_a & z_a \\
      x_b & y_b & z_b \\
      x_c & y_c & z_c 
    \end{vmatrix}
\end{gather*}

Т.е. \[
\vec{a} \vec{b} \vec{c} =
\begin{vmatrix}
  x_a & y_a & z_c \\
  x_b & y_b & z_b \\
  x_c & y_c & z_c \\
\end{vmatrix}
\]
\subsubsection{Геометрическое приложение смешанного произведения}

Пусть $\vec{a}, \vec{b}, \vec{c}$. Совместим начала этих векторов и достроим до параллелипипеда. Тогда $V_{paral} = |\vec{a} \vec{b} \vec{c}|$.
% Рисуночек 
\begin{note}
  \[
    V_{pyramid} = \frac{1}{6} V_{paral} = \frac{1}{6} \cdot |\vec{a} \vec{b} \vec{c}| 
  \] 
\end{note}


\section{Прямая на плоскости}

\subsection{Каноническое уравнение}

Пусть прямая $l$ проходит через точку $M_0(x_0, y_0)$ и задана направляющим вектором $\vec{S} = \{m, n\}$ (т.е. вектор паралеллен прямой). Выберем на прямой $l$ произвольную точку $M$.
Составим $\overrightarrow{M_0M} = \{x - x_0, y - y_0, z - z_0\}$. 
\[
  \overrightarrow{M_0M} \parallel \vec{s} \implies \boxed{\frac{x - x_0}{m} = \frac{y - y_0}{n}}
\]

\subsection{Параметрическое уравнение}

Пусть прямая $l$ задана каноническим уравнением: \[
\frac{x - x_0}{m} = \frac{y - y_0}{n}
\] 
Обозначим коеффициент пропорциональности через $t$. Тогда:
\begin{gather*}
  \begin{matrix}
    \frac{x - x_0}{m} = t \\
    \frac{y - y_0}{n} = t
  \end{matrix} \implies \boxed{\begin{cases}
    x = x_0 + mt \\
    y = y_0 + nt
  \end{cases}}
\end{gather*}
 
\subsection{Через две точки}

Пусть прямая $l$ проходит через точки $M_0(x_0, y_0)$ и $M(x, y)$. Выберем на прямой $l$ произвольную точку $M_1(x_1, y_1)$. Составим два вектора $\overrightarrow{M_0M}$, $\overrightarrow{M_0, M_1}$.
\begin{gather*}
  \overrightarrow{M_0M} = \{x - x_0, y - y_0\} \\
  \overrightarrow{M_0M_1} = \{x_1 - x_0, y_1 - y_0\}
\end{gather*}

Т.к. вектора коллинеарны, то и соответствующие координаты пропорциональны:
 \[
  \boxed{\frac{x - x_0}{x_1 - x_0} = \frac{y - y_0}{y_1 - y_0}}
\]

\subsection{В отрезках}

% Рисуночек
Пусть прямая $l$ отсекает от координатного угла отрезки $a$ и $b$. Тогда прямая $l$ проходит через точки $A(0, a)$ и  $B(b, 0)$.
\begin{gather*}
  \frac{x - a}{0 - a} = \frac{y - 0}{b - 0} \implies \boxed{\frac{x}{a} + \frac{y}{b} = 1}
\end{gather*}


