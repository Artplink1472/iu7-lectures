\section{Представление графической информации}

Существует 2 подхода к представлению изображения:
\begin{itemize}
  \item растровая графика
  \item векторная графика
\end{itemize}

\begin{definition}
  \textbf{Растровое изображение} - изображение, формирующееся в виде матрицы пикселей различного цвета.
\end{definition}

\begin{definition}
  \textbf{Растр} - матрица пикселей.
\end{definition}

\begin{definition}
  \textbf{Разрешающая способность} - расстояние между соседними пикселями. Измеряется количеством пикселей на единицу длины.
\end{definition}

\begin{itemize}
  \item \textbf{DPI} - кол-во пикселей на дюйм.
  \item \textbf{Размер растра} - измеряется в кол-вом пикселей по горизонтали и по вертикали. 
\end{itemize}

\subsection{Цвет}

\textbf{Глубина цвета}
\begin{itemize}
  \item Двухцветные -- 1 бит на пиксель 
  \item Полутоновые -- градации цвета
  \item Цветные
\end{itemize}

\textbf{Различные цветовые модели} 
\begin{itemize}
  \item RGB -- red green blue 
  \item 
\end{itemize}

