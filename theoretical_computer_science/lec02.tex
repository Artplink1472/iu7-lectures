\lesson{2}{Программирование}

\section{Системы исчисления}
\subsection{Виды систем исчисления}

\begin{definition}
  \textbf{Система исчисления} - совокупность приёмов и правил для записи чисел цифровыми знаками.
\end{definition}

\begin{definition}
  Символы, используемые в любой системе исчисления, называются \textbf{цифрами}.
\end{definition}

\begin{definition}
  Совокупность цифр для записи чисел называется \textbf{алфавитом}.
\end{definition}

\subsubsection{Непозиционные системы исчисления}

\begin{definition}
  Если в системе счисления каждой цифре в любом месте числа соответствует одно и то же значение, то такая система нахвается \textbf{непозиционной}.
\end{definition}
\begin{eg}
  \textit{Римская система} - с некоторыми докущениями 
\end{eg}

\subsubsection*{Римские числа}

Значение цифры не зависит от её местоположения.
\begin{itemize}
  \item Если цифра с меньшим значение стоит слева от цифры с большим значением, то её знак "минус".
  \item Если цифра с меньшиими значением стоит справа от цифры с большиим значением, то её знак "плюс"
  \item Вычитать из $10^n$ можно только один раз, не перепрыгивая через разряды.
\end{itemize}

Недостатки непозиционных систем исчисления:
\begin{itemize}
  \item Трудность записи больших чисел
  \item Трудность выполнения арифметических операций
\end{itemize}

\subsubsection{Позиционные системы исчисления}

\begin{definition}
  Система исчисления называется \textbf{позиционной}, если одна и та же цифра имеет различное значение, которое определяется её позицией в последовательности цифр, обозначающей запись числа.
  \[
    \overline{x_{n} x_{n-1} \ldots x_0} = x_{n} q_{n} + x_{n-1} q_{n-1} + \ldots + x_0 q_0 \text{, где}
  \]
  \begin{center}
    $x_{n}, x_{n-1}, \ldots, x_{0}$ - символы, обозначающие целые числа; \\
    $q_{n}, q_{n-1}, \ldots, q_0$ - веса.
  \end{center}
\end{definition}

\begin{definition}
  Номер позиции, котрой определяет вес цифры, расположенной на этой позиции, называется \textbf{разярдом}.
\end{definition}

Особый интерес представляют системы исчисления, в которых веса цифры - геометрическая прогрессия со знаменателем q. \\
Тогда число имеет вид:
\[
  x_q = \sum_{i=-m}^{i=m} x_i q^i
\] 

\begin{definition}
  \textbf{Основание} q \textbf{базис} позиционной системы исчисления - количество знаков или символов, используемых для отображения числа в данной системе.
\end{definition}


\subsubsection{Перевод чисел из одной системы счисления в другую}

Алгоритм перевода состоит из двух этапов:
\begin{enumerate}
  \item Последовательное деление целой части и образующихся целыъ частных на основание новой системы счисления.
  \item Последовательное умножение дробной части и дробных частей, получающихся произведений на то же основание новой системы счисления, \textit{записанное цифрами исходной системы счисления}. 
\end{enumerate}


\begin{table}[htpb]
  \centering
  \caption{Значения чисел в различных системах счисления}
  \label{tab:label}
  \begin{tabular}{ | c | c | c | c | }
    \hline
    bin & oct/hex & bin & hex \\
    \hline
    0000 & 0 & 1000 & 8 \\
    0001 & 1 & 1001 & 9 \\
    0010 & 2 & 1010 & A \\
    0011 & 3 & 1011 & B \\
    0100 & 4 & 1100 & C \\
    0101 & 5 & 1101 & D \\
    0110 & 6 & 1110 & E \\
    0111 & 7 & 1111 & F \\
    \hline
  \end{tabular}
\end{table}

